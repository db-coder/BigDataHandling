\begin{frame}
\titlepage
\end{frame}

\begin{frame}
\frametitle{\centerline{CS 215 Project \newline A CS 215 Report by Group 01}}
\begin{columns}[T]
\begin{column}{\textwidth}
\begin{block}{Working}
1.Introduction \newline \pause 
2.Working \newline \pause
3.Conclusion
\end{block}
\end{column}
\end{columns}
\end{frame}

\begin{frame}
\frametitle{\centerline{Introduction}}
This report is created as a part of our CS 215 project to easily convey the internal working of the project.\newline
In the coming slides, you will find a detailed description of the project.
Happy Reading!
\end{frame}

\begin{frame}
\frametitle{\centerline{Working}}
\begin{enumerate}
\item Defined classes 'Attribute', 'Country' and 'Sentence' along with their data members and data functions to help organise and simplify the code. \pause
\item Defined three dictionaries (which are basically maps): \pause
\setbeamertemplate{enumerate items}[default]
\begin{enumerate}
\item cc : A dictionary that maps a country code to the corresponding index number, in the order in which they are received the first time. \pause
\item cn : A dictionary that maps a country name to the corresponding index number, in the order in which they are received the first time. \pause
\item p  : A dictionary that maps an attribute to the corresponding index number, in the order in which they are received the first time. \pause
\end{enumerate}
\end{enumerate}
\end{frame}

\begin{frame}
\frametitle{\centerline{Working}}
\begin{enumerate}
\item On running the code, main function is called by default, which: \pause
\setbeamertemplate{enumerate items}[default]
\begin{enumerate}
\item Reads from "countries\_idmap.txt" and populates 'country\_array' \pause
\item Reads from "selected\_indicators" and populates "attributes\_array\_main" \pause
\item Runs a 'for' loop that adds this attribute-list to each of the countries already listed above. \pause
\item Reads from "kb-facts-train\_SI.csv" and fills up the attribute-list for each country using this data. \pause
\item Calls regression() function for each of the countries, which: \pause
\setbeamertemplate{enumerate items}[default]
\begin{enumerate}
\item Finds the sample mean and variance for each attribute of the country \pause
\item Uses linear regression and finds the regression coefficients \pause
\end{enumerate}
\end{enumerate}
\end{enumerate}
\end{frame}

\begin{frame}
\frametitle{\centerline{Working}}
\begin{enumerate}
\item On running the code, main function is called by default, which: \pause
\setbeamertemplate{enumerate items}[default]
\begin{enumerate}
\item Reads "sentences.tsv" and converts each row of input to a corresponding Sentence class object and calls the function doAll() on that Sentence object. \pause
\setbeamertemplate{enumerate items}[default]
\begin{enumerate}
\item The doAll() function uses the country-name passed with the Sentence object and goes on removing the last letter until it finds a match with one of the countries in the country\_array list (or the country-name becomes empty) \pause
\item For calculation of score, the function uses the normlised exponential formula $$ score = 100 * \exp{-(x-u)}^{2} / (2(s^{2}) ) / (\sqrt(2*\pi)*s) ,$$ s being the standard deviation of the sample values of attribute in consideration.
\end{enumerate}
\end{enumerate}
\end{enumerate}
\end{frame}

\begin{frame}
\frametitle{\centerline{Conclusions}}
This is our project as a part of the CS 215 course. Hope you enjoyed going through the report as much as we enjoyed making it.
Any other reviews and/or suggetions are most welcome. Looking forward to hearing from you.
Thank You :)
\end{frame}